\documentclass{article}
\usepackage[T1]{fontenc}
\usepackage[utf8]{inputenc}
\usepackage{datetime}

\newdate{date}{20}{11}{2019}
\date{\displaydate{date}}

\title{Empirical Asset Pricing via Machine Learning: Summary}
\author{Jan Weidemüller \and Jens Kienle \and Mohammed Bayoudh \and Pierre Brosemer \and Rusheel Iyer \and Tobias Huber}

\begin{document}
	
	\maketitle
	\pagebreak
	
	The authors of the article compare and evaluate various machine learning methods in how successfully they are able to predict equity risk premia. The aim of the paper is not only to justify the economic benefit of using machine learning in predicting equity returns, but to additionally find the major variables that can aid in explaining the behaviour of the risk premia.
	
	Furthermore, the authors clarify the suitability of machine learning to this task through its general affinity towards prediction tasks, the ability to eliminate highly correlated variables, as well as its aptitude for problems with obscure form. The limitation of machine learning is also recognized as its inability to autonomously determine deep-seated relations between asset prices and conditioning factors.
\end{document}