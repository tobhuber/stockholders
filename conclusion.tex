\section{Conclusion}

In this paper the authors try to make two points clear. The first being that machine learning can help a lot in the prediction and measurement of risk premia 
and secondly, that these are reliable.
In order to achieve these goals a multitude of techniques and models are analyzed and compared. To further understand the factors affecting the predictions, each variables importance was also assessed.
A significant conclusion that can be drawn from the paper is the ability of machine learning to make accurate predictions on a long and short timeline.
Furthermore it was proved that the most influential characteristics were unaffected by noise, thus confirming the
robustness of the results returned.
Throughout the study, non-linear models consistently outperform linear models. More particularly, the 3-layered neural network is found to be the best overall performer. This also confirmed a preliminary question regarding the superior performance of shallow neural networks vs. deeper neural networks.
\newline\noindent
All the findings presented in this paper solidify the potential of machine learning in the field of predicting equity risk premia. This proves promising for the future of financial research and economic prediction.